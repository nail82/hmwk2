\title{CS690 Homework 2}
\author{Ted Satcher}
\documentclass[12pt]{article}
\usepackage{cite}
\date{\today}
\begin{document}
\maketitle

\section*{1. Comparison of Grids and Clusters}
Cluster and grid based computing paradigms have a number of
similarities and several important differences. Both are a means of
utilizing surplus resources from a collection of computing
hardware. Below, I will attempt to identify some distinguishing
characteristics of the two approaches.

% admin domains
Clusters consist of a group of networked computers, brought together
on a permanent or semi-permanent basis to solve computationally
expensive problems.  Normally, they are restricted to a single
administrative domain.

Grids are also networked computer resources, but they vary from
clusters in that the resources span administrative boundaries. This
spanning of boundaries is called a Virtual Organization and is a
defining characteristic of grid computing~\cite{classtext}
and~\cite{anatomy}.

% architectures
% - cluster: master/slave, hadoop as example
%   homogeneous
% - grid: usually no centralized control, since crossing boundaries
%   doesn't have to be computation. can be other resources like
%   storage
%   heterogeneous
A second distinction between clusters and grids is the architecture of
the network and the software environment present on the nodes of the
system. Grid members can be heterogeneous in hardware and software,
while clustered systems are homogeneous in nature with each node
executing the same or similar local operating system~\cite{classtext}.

Clusters are often constructed in a master/worker layout. An example
of this cluster organization is the popular Hadoop cluster framework
used widely in industry~\cite{hadoop}.

Grids vary in their network layout and are not normally subject to
centralized control.  Also, the resources made available by a grid may
not be computational in nature. Some grids are organized primarily for
data sharing~\cite{anatomy}.

% applications
% - cluster Long running batch applications, mostly within 1 org
% - grid Collaborative problem solving across organizations
%   multi-organizational scientific research
Finally, clusters and grids vary based on the computation performed.
Clusters are often used to perform long running calculations on
unreliable resources~\cite{hadoop},~\cite{condor-man}. This execution
model requires the scheduling mechanism to be aware of failures in
order to restart failed portions of a computation.

Grid systems, on the other hand, are targeted to resource
sharing. There can be a high performance computation component to a
grid architecture, but this is not necessarily the focus of a
grid~\cite{anatomy}.




\section*{2. HTCondor}
\subsection*{Part a}
HTCondor uses the idle resources of machines in the worker pool and
was designed to be sensitive to the owners of worker pool
machines~\cite{condor-man}.  There is nothing preventing an
administrator from setting up a worker pool of dedicated machines to a
Condor cluster, however.

\subsection*{Part b}
High Performance Computing (HPC) is a computation strategy using a
tightly coupled machine or network to attack difficult computations
with short time horizons. Applications requiring HPC computations
would normally be run at a super computer installation.  An example
application is weather modeling software~\cite{hpcwire}.

High Throughput Computing (HTC) is used to reliably execute long
running computations on more mainstream, commodity hardware.  These
long running programs must be executed on systems that monitor the
intermediate calculations and provide rollback and restart in the face
of hardware failures.  Example applications include Monte Carlo
simulation runs and studies that explore the parameter space of a
problem~\cite{hpcwire}.

\subsection*{Part c}
The ClassAd mechanism in HTCondor is used to match hardware resources
with submitted jobs.  Workers in the Condor pool publish attributes of
their runtime environment (e.g. CPU, memory, current workload). Job
submitters present preferences for the type of hardware they would
like to bring to bear on their problem.  The Condor scheduler monitors
both streams of information and pairs up submitted jobs with available
execution resources~\cite{condor-man}.

\section*{3. Multiple Choice}
3a iii
3b i
3c i
3d ii
3e ii

\bibliographystyle{IEEEtran}
\bibliography{IEEEabrv,hmwk2}

\end{document}